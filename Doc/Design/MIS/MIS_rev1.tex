\documentclass[12,english]{article}
\usepackage[letterpaper, portrait, margin=1in]{geometry}

\usepackage{amsmath}
\usepackage[T1]{fontenc}
\usepackage{babel}
\usepackage{textcomp}
\usepackage{titlesec}
\setcounter{secnumdepth}{4}
\usepackage{hyperref}
\usepackage[normalem]{ulem}
\usepackage{xcolor}
\usepackage{booktabs}
\usepackage{placeins}
\usepackage{multirow}
\usepackage[dvipsnames]{xcolor}
\usepackage{soul}

\usepackage[usenames, dvipsnames]{color}

\hypersetup{
    bookmarks=true,         % show bookmarks bar?
      colorlinks=true,       % false: boxed links; true: colored links
    linkcolor=black,          % color of internal links (change box color with linkbordercolor)
    citecolor=green,        % color of links to bibliography
    filecolor=magenta,      % color of file links
    urlcolor=cyan           % color of external links
}

\titleformat{\paragraph}
{\normalfont\normalsize\bfseries}{\theparagraph}{1em}{}
\titlespacing*{\paragraph}
{0pt}{3.25ex plus 1ex minus .2ex}{1.5ex plus .2ex}

\title{SE 3XA3: Modulule Interface Specification\\Plagiarism Check}

\author{Team 310, MXQ Gang
        \\ Mrinal Tiwari, tiwarim
		\\ Qifeng Xu, xuq14
		\\ Xu Wang, wangx147
}


\date{\today}

\begin{document}
\maketitle
\newpage
\tableofcontents
\newpage




		%%%%%%%%%%%%%%%%%%%%%%%%%%%%%%%%

	
\section{MIS of \boldtf{System Utilities}}
		\subsection{Interface Syntax}
			\subsubsection{Exported Access Programs}
				\begin{tabular}[pos]{|c|c|c|c|}
					
					\hline
					%	\label
					\textbf{Name}& \textbf{In} & \textbf{Out} & \textbf{Exceptions} \\ \hline
					UserExists & string & boolean & - \\ \hline
					VerifyPW & string, string & boolean & -\\ \hline
					CountTokens & string & integer & -\\ \hline
					
				\end{tabular}
				
		\subsection{Interface Semantics}
			\subsubsection{State Variables}
			Not applicable
			
			\subsubsection{Environmental Variables}
			Not Applicable
			
			\subsubsection{Assumptions}
        `   There will be no exceptions raised in this module as Utils module handles the exceptions when calling this module.
        
			\subsubsection{Access Program Semantics}
			UserExists(username):
			
			Input: Username as string
			
			Transition: Searches for the username in the database
			
			Output: Boolean True or False
			
			Exception: none\\
			\\
			
			VerifyPW(username, password):
			
			Input: Username as string, password as string
			
			Transition: Compares the hashed password stored in the database to the password given as input
			
			Output: Boolean True or False
			
			Exception: none\\
			\\
			
			CountTokens(username):
			
			Input: Username as string
			
			Transition: Searches for number of tokens stored in the database for that user.
			
			Output: Returns the number of tokens as integer
			
			Exception: none\\
			\\
			
			
		%%%%%%%%%%%%%%%%%%%%%%%%%%%%%%%%%%	
			
			
\section{MIS of \boldtf{Register Resource}}
		\subsection{Interface Syntax}
			\subsubsection{Exported Access Programs}
				\begin{tabular}[pos]{|c|c|c|c|c|}
					
					\hline
					%	\label
					\textbf{Name}& \textbf{URL} & \textbf{In} & \textbf{Out} & \textbf{Exceptions} \\ \hline
					post & \slash register & json & json & 301 \\ \hline
					
				\end{tabular}
				
		\subsection{Interface Semantics}
			\subsubsection{State Variables}
			namepass: Json input containing the name and password of the user to register.  \\*
			name: user's name as string extracted from the Json.   \\*
			password: user's password as string extracted from the Json    \\*
			
			\subsubsection{Environmental Variables}
			Not Applicable
			
			\subsubsection{Assumptions}
           It is assumed that user would not send a invalid json like json containing invalid number of inputs.
        
			\subsubsection{Access Program Semantics}
			post(Resource):
			
			Input: Resource as json that is passed over the register URL of Restful API
			
			Transition: Updates the database with the new user credential
			
			Output: retJson as json that contains the reply message and status code
			
			Exception: 301 - user already exists\\
			\\
			
			
			%%%%%%%%%%%%%%%%%%%%%%%%%%%%%%%%%%	
			
			
\section{MIS of \boldtf{Detect Resource}}
		\subsection{Interface Syntax}
			\subsubsection{Exported Access Programs}
				\begin{tabular}[pos]{|c|c|c|c|c|}
					
					\hline
					%	\label
					\textbf{Name}& \textbf{URL} & \textbf{In} & \textbf{Out} & \textbf{Exceptions} \\ \hline
					post & \slash detect & json & json & 301, 302, 303 \\ \hline
					
				\end{tabular}
				
		\subsection{Interface Semantics}
			\subsubsection{State Variables}
			namepassref : Json input containing username, admin password and refill amount. \\* 
			name: user's name as string extracted from the Json.   \\*
			\textcolor{red}{password: User's password as string extracted from the Json} \\*
			\textcolor{red}{text1: First text as string extracted from the json} \\*
			\textcolor{red}{text2: Second text as string extracted from the json} \\*
			\st{admin\_password: admin password as string extracted from the Json }   \\*
			\st{refill\_amt: the amount of tokens to refill as integer extracted from the Json}  \\*
			\subsubsection{Environmental Variables}
			Not Applicable
			
			\subsubsection{Assumptions}
           It is assumed that user would not send a invalid json containing invalid number or type of inputs.
        
			\subsubsection{Access Program Semantics}
			post(Resource):
			
			Input: Resource as json that is passed over the refill URL of Restful API
			
			Transition: Adds the refill amount to the existing number of tokens
			
			Output: retJson as json that contains the reply message and status code
			
			Exception: 301 - invalid user, 302 - invalid password, 303 - out of tokens\\
			\\
			
			%%%%%%%%%%%%%%%%%%%%%%%%%%%%%%%%%%	
			
	\section{MIS of \boldtf{Refill Resource}}
		\subsection{Interface Syntax}
			\subsubsection{Exported Access Programs}
				\begin{tabular}[pos]{|c|c|c|c|c|}
					
					\hline
					%	\label
					\textbf{Name}& \textbf{URL} & \textbf{In} & \textbf{Out} & \textbf{Exceptions} \\ \hline
					post & \slash refill & json & json & 301,302, 304 \\ \hline
					
				\end{tabular}
				
		\subsection{Interface Semantics}
			\subsubsection{State Variables}
			namepass: Json input containing the name and password of the user to register. \\*
			\textcolor{red}{admin\_password: admin password as string extracted from the Json}.  \\*
			\textcolor{red}{refill\_amt: the amount of tokens to refill as integer extracted from the Json}. 
			
			\subsubsection{Environmental Variables}
			Not Applicable
			
			\subsubsection{Assumptions}
           It is assumed that user would not send a invalid json like json containing invalid number of inputs.
        
			\subsubsection{Access Program Semantics}
			post(Resource):
			
			Input: Resource as json that is passed over the register URL of Restful API
			
			Transition: Updates the database with the new user credential
			
			Output: retJson as json that contains the reply message and status code
			
			Exception: 301 - invalid user, 302 - invalid password, 304 - wrong admin password\\
			\\
			
			%%%%%%%%%%%%%%%%%%%%%%%%%%%%%%%%%%	
			
			
	\section{MIS of \boldtf{Integration Module}}
		\subsection{Interface Syntax}
			\subsubsection{Exported Access Programs}
				\begin{tabular}[pos]{|c|c|c|c|}
					
					\hline
					%	\label
					\textbf{Name} & \textbf{In} & \textbf{Out} & \textbf{Exceptions} \\ \hline
					
					\textcolor{red}{createCorsrequest} & \textcolor{red}{method}, \textcolor{red}{url} & \textcolor{red}{CorsRequest} & \textcolor{red}{Error 404} \\ \hline
					\textcolor{red}{makeCorsRequestRegister} & \textcolor{red}{string, string} & \textcolor{red}{-} &  \textcolor{red}{Error 500} \\ \hline
					\textcolor{red}{Registersend} & \textcolor{red}{-} & \textcolor{red}{-} &  \textcolor{red}{-} \\ \hline
					\textcolor{red}{makeCorsRequestCheck} & \textcolor{red}{string, string, string, string} & \textcolor{red}{-} &  \textcolor{red}{Error 500} \\ \hline
					\textcolor{red}{Checksend} & \textcolor{red}{-} & \textcolor{red}{-} &  \textcolor{red}{-} \\ \hline
					\textcolor{red}{makeCorsRequestRefill} & \textcolor{red}{string, string, int} & \textcolor{red}{-} &  \textcolor{red}{Error 500} \\ \hline
					\textcolor{red}{Refillsend} & \textcolor{red}{-} & \textcolor{red}{-} &  \textcolor{red}{-} \\ \hline
					
					
				\end{tabular}
				
		\subsection{Interface Semantics}
			\subsubsection{State Variables}
			Not applicable
			
			\subsubsection{Environmental Variables}
			\st{Not Applicable} \\*
			\textcolor{red}{username : stores username entered by user in the textbox using keyboard}  \\*
			\textcolor{red}{password : stores password entered by user in the textbox using keyboard}  \\*
			\textcolor{red}{text1 : stores first text entered by user in the textbox using keyboard}  \\*
			\textcolor{red}{text2 : stores second entered by user in the textbox using keyboard}  \\*
			\textcolor{red}{adminPassword : stores admin password entered by user in the textbox using keyboard}  \\*
			\textcolor{red}{Submit : responds to click made by user}  \\*
			
			\subsubsection{Assumptions}
			\textcolor{red}{It is assumed that the backend is running and is properly connected through Amazon Web Services.} \\*
           \st{There will be no exceptions raised in this module as Utils module handles the exceptions when calling this module.} 
        
        
			\subsubsection{Access Program Semantics}
			\st{
			UserExists(username):

			Input: Username as string
			
			Transition: Searches for the username in the database
			
			Output: Boolean True or False
			
			Exception: none\\
			\\
			
			VerifyPW(username, password):
			
			Input: Username as string, password as string
			
			Transition: Compares the hashed password stored in the database to the password given as input
			
			Output: Boolean True or False
			
			Exception: none\\
			\\
			
			CountTokens(username):
			
			Input: Username as string
			
			Transition: Searches for number of tokens stored in the database for that user.
			
			Output: Returns the number of tokens as integer
			
			Exception: none\\ }
			
			\textcolor{red}{\textbf{createCorsRequest}(method, url):} \\*
			\textcolor{red}{Input: method in terms of GET or POSt, url as URL of the REST API} \\*
			\textcolor{red}{Transition: creates a HTTP Cors request of the url on the preferred method} \\*
			\textcolor{red}{ Output: Returns the Cors request } \\*
			\textcolor{red}{Exception: Error 404 URL not found } 
			\\
			
			\textcolor{red}{\textbf{makeCorsRequestRegister}(uname, pass):} \\*
			\textcolor{red}{Input: username as string, password as string} \\*
			\textcolor{red}{Transition: Makes a HTTP request call of the url with username and password} \\*
			\textcolor{red}{ Output: - } \\*
			\textcolor{red}{Exception: Error 500 Internal server error } 
			\\
			
			\textcolor{red}{\textbf{RegisterSend}():} \\*
			\textcolor{red}{Input: -} \\*
			\textcolor{red}{Transition: accepts the username and password from the text box} \\*
			\textcolor{red}{ Output: - } \\*
			\textcolor{red}{Exception:-  } 
			\\
			
			\textcolor{red}{\textbf{makeCorsRequestCheck}(uname, pass, text1, text2):} \\*
			\textcolor{red}{Input: username as string, password as string, text1 as string, text2 as string} \\*
			\textcolor{red}{Transition: Makes a HTTP request call of the url with username, password, text and text2} \\*
			\textcolor{red}{ Output: - } \\*
			\textcolor{red}{Exception: Error 500 Internal server error } 
			\\
			
			\textcolor{red}{\textbf{CheckSend}():} \\*
			\textcolor{red}{Input: -} \\*
			\textcolor{red}{Transition: accepts the username, password, text1 and text2 from the text box} \\*
			\textcolor{red}{ Output: - } \\*
			\textcolor{red}{Exception:-  } 
			\\
			
			\textcolor{red}{\textbf{makeCorsRequestRefill}(uname, adminPass, refillAmt):} \\*
			\textcolor{red}{Input: username as string, admin password as string, refill amount as int} \\*
			\textcolor{red}{Transition: Makes a HTTP request call of the url with username, admin password and refill amount} \\*
			\textcolor{red}{ Output: - } \\*
			\textcolor{red}{Exception: Error 500 Internal server error } 
			\\
			
			\textcolor{red}{\textbf{RefillSend}():} \\*
			\textcolor{red}{Input: -} \\*
			\textcolor{red}{Transition: accepts the username, admin password and refill amount from the text box} \\*
			\textcolor{red}{ Output: - } \\*
			\textcolor{red}{Exception:-  } 
			
			\\
			
			%%%%%%%%%%%%%%%%%%%%%%%%%%%%%%%%%%	
			
			
			
	\newpage	
	\section{Major Revision History}
\begin{table}[!htbp]
	\begin{tabular}{|l|l|}
		\toprule
		Date & Revision\\ \hline
		February 3, 2020 & Rough draft of sections\\ \hline
		February 15, 2019 & Revised sections \\ \hline
		February 16, 2019 & Revision 0 complete\\ \hline
		March 12, 2020 & MIS, Module Guide, Requirements Document, Test Plan, and Test Report revision 0 \\ \hline
		\textcolor{red}{April 6, 2020} & \textcolor{red}{Updated sections, updated Integration Module} \\ \hline
		
	\end{tabular}
\end{table}
			


			
\end{document}
